%%%%%%%%%%%%%%%%%
% This is an example CV created using altacv.cls (v1.1.5, 1 December 2018) written by
% LianTze Lim (liantze@gmail.com), based on the
% Cv created by BusinessInsider at http://www.businessinsider.my/a-sample-resume-for-marissa-mayer-2016-7/?r=US&IR=T
%
%% It may be distributed and/or modified under the
%% conditions of the LaTeX Project Public License, either version 1.3
%% of this license or (at your option) any later version.
%% The latest version of this license is in
%%    http://www.latex-project.org/lppl.txt
%% and version 1.3 or later is part of all distributions of LaTeX
%% version 2003/12/01 or later.
%%%%%%%%%%%%%%%%

%% If you are using \orcid or academicons
%% icons, make sure you have the academicons
%% option here, and compile with XeLaTeX
%% or LuaLaTeX.
% \documentclass[10pt,a4paper,academicons]{altacv}

%% Use the "normalphoto" option if you want a normal photo instead of cropped to a circle
% \documentclass[10pt,a4paper,normalphoto]{altacv}

\documentclass[10pt,a4paper,ragged2e]{altacv}

%% AltaCV uses the fontawesome and academicon fonts
%% and packages.
%% See texdoc.net/pkg/fontawecome and http://texdoc.net/pkg/academicons for full list of symbols. You MUST compile with XeLaTeX or LuaLaTeX if you want to use academicons.

% Change the page layout if you need to
\geometry{left=2cm,right=10cm,marginparwidth=6.8cm,marginparsep=1.2cm,top=1.25cm,bottom=1.25cm}

% Change the font if you want to, depending on whether
% you're using pdflatex or xelatex/lualatex
\ifxetexorluatex
  % If using xelatex or lualatex:
  \setmainfont{Carlito}
\else
  % If using pdflatex:
  \usepackage[utf8]{inputenc}
  \usepackage[T1]{fontenc}
  \usepackage[default]{lato}
\fi
\usepackage{hyperref}

% Change the colours if you want to
\definecolor{VividPurple}{HTML}{247392}
\definecolor{SlateGrey}{HTML}{2E2E2E}
\definecolor{LightBlue}{HTML}{03A9F4}
\definecolor{LightGrey}{HTML}{2E2E2E}
\colorlet{heading}{VividPurple}
\colorlet{accent}{VividPurple}
\colorlet{emphasis}{SlateGrey}
\colorlet{body}{LightGrey}

\hypersetup{
  colorlinks=true,
  linkcolor=LightBlue,
  urlcolor=LightBlue
}

% Change the bullets for itemize and rating marker
% for \cvskill if you want to
\renewcommand{\itemmarker}{{\small\textbullet}}
\renewcommand{\ratingmarker}{\faCircle}

%% sample.bib contains your publications
\addbibresource{sample.bib}

\begin{document}
\name{Himanshu Soni}
\tagline{Software Engineer}
% Cropped to square from https://en.wikipedia.org/wiki/Marissa_Mayer#/media/File:Marissa_Mayer_May_2014_(cropped).jpg, CC-BY 2.0
%\photo{3.3cm}{profile.jpg}
\personalinfo{%
  % Not all of these are required!
  % You can add your own with \printinfo{symbol}{detail}
  \email{er.hsoni92@gmail.com}
  \phone{(+91)7382978592}
%  \mailaddress{Address, Street, 00000 County}
  \location{Hyderabad, Telangana,India}
  \homepage{hdesign.in}
%  \twitter{@marissamayer}
   \github{github.com/hsoni92} % I'm just making this up though.
%   \orcid{orcid.org/0000-0000-0000-0000} % Obviously making this up too. If you want to use this field (and also other academicons symbols), add "academicons" option to \documentclass{altacv}
}

%% Make the header extend all the way to the right, if you want.
\begin{fullwidth}
\makecvheader
\end{fullwidth}

%% Depending on your tastes, you may want to make fonts of itemize environments slightly smaller
\AtBeginEnvironment{itemize}{\small}

%% Provide the file name containing the sidebar contents as an optional parameter to \cvsection.
%% You can always just use \marginpar{...} if you do
%% not need to align the top of the contents to any
%% \cvsection title in the "main" bar.
\cvsection[page1sidebar]{Experience}

\cvevent{Senior Software Engineer}{One Convergence}{March 2021 -- Present}{Hyderabad, Telangana, India}
\cvevent{Software Engineer}{One Convergence}{August 2018 -- March 2021}{Hyderabad, Telangana, India}
\begin{itemize}
\item Responsible for planning, development, integration and testing for various UI based projects at the Organization. Developing and maintaining automation test suits.

\end{itemize}

\divider

\cvevent{Full Stack Developer}{Freelancing}{Mar 2013 -- May 2018}{New Delhi/Hyderabad, India}
\begin{itemize}
\item  Prior to completing my education, I have worked on a multitude of web based projects for a range of clients providing UX (Photoshop, Illustrator, XD), Architecture and Development (ReactJS, PHP, HTML, CSS/LESS, jQuery, Responsive Layouts).

\end{itemize}

%\divider

\cvsection{SKILLS}

\cvskill{NodeJS, ReactJS, Webpack}{5}
\cvskill{HTML, CSS/LESS, JavaScript, JQuery, PHP, SQL}{5}
\cvskill{Jest, Jasmine, Selenium, Puppeteer}{4}
\cvskill{AWS, Kubernetes, Docker}{3}
\cvskill{Shell Scripting}{5}
\cvskill{Python, GraphQL}{3}
\cvskill{Adobe XD, Figma}{3}
\cvskill{Adobe Photshop, Adobe Illustrator}{5}
\cvskill{GitHub, Postman}{4}

% \divider


\cvsection{Projects}
\smallskip
\begin{itemize}
\item \textbf{Virtual Infrastructure Web App}
\newline
A UI Project comprising of Networking and Virtualization workflows built in ReactJS, BackboneJS, NodeJS. | 2019 - Present
\smallskip
\item \textbf{Automation for UI/Windows/REST API}
\newline
Automation using python-selenium, python based propriety framework and Windows Powershell | Nov. 2018 - Oct. 2019
\smallskip
\item \textbf{Deep Learning Web App}
\newline
A Deep Learning Platform for Data Scientists for running/managing datasets and models built with ReactJS | Mar. 2018 - Nov. 2018
\smallskip
\item \textbf{Lazarus Deployer (Self Indulged Project)}
\newline
A one click shared hosting migration WebApp. Built using ReactJS, NodeJS, PHP, REST API, Telegram Bot API, Dropbox API | Mar. 2021 - PRESENT
\smallskip
\item \textbf{Artemis Cluster (Self Indulged Project)}
\newline
A Dropbox Cluster Manager for deploying n-number of dropbox syncs in a linux based environment. Built using ReactJS, NodeJS, Docker , REST API and Dropbox API | Feb. 2019 - PRESENT
\smallskip
\item  \textbf{Automated Compute Infrastructure for Organizations using\newline OpenStack}
\newline
A WebApp built on top of OpenStack using python and shell as a middleware. Built with PHP, JQuery and python-openstack-sdk |  Jan. 2018 - Jun. 2018
\end{itemize}


% \cvevent{Product Engineer}{Google}{23 June 1999 -- 2001}{Palo Alto, CA}
% \begin{itemize}
% \item Joined the company as employe \#20 and female employee \#1
% \item Developed targeted advertisement in order to use user's search queries and show them related ads
% \end{itemize}

%\cvsection{A Day of My Life}

% Adapted from @Jake's answer from http://tex.stackexchange.com/a/82729/226
% \wheelchart{outer radius}{inner radius}{
% comma-separated list of value/text width/color/detail}
% Some ad-hoc tweaking to adjust the labels so that they don't overlap
% \wheelchart{1.5cm}{0.5cm}{%
%   10/10em/accent!30/Sleeping \& dreaming about work,
%   25/9em/accent!60/Public resolving issues with Yahoo!\ investors,
%   5/13em/accent!10/\footnotesize\\[1ex]New York \& San Francisco Ballet Jawbone board member,
%   20/15em/accent!40/Spending time with family,
%   5/8em/accent!20/\footnotesize Business development for Yahoo!\ after the Verizon acquisition,
%   30/9em/accent/Showing Yahoo!\ employees that their work has meaning,
%   5/8em/accent!20/Baking cupcakes
% }

\clearpage

% \cvsection[page2sidebar]{Publications}

\nocite{*}

% \printbibliography[heading=pubtype,title={\printinfo{\faBook}{Books}},type=book]

% \divider

% \printbibliography[heading=pubtype,title={\printinfo{\faFileTextO}{Journal Articles}}, type=article]

% \divider

% \printbibliography[heading=pubtype,title={\printinfo{\faGroup}{Conference Proceedings}},type=inproceedings]

% %% If the NEXT page doesn't start with a \cvsection but you'd
% %% still like to add a sidebar, then use this command on THIS
% %% page to add it. The optional argument lets you pull up the
% %% sidebar a bit so that it looks aligned with the top of the
% %% main column.
% % \addnextpagesidebar[-1ex]{page3sidebar}


\end{document}
